\documentclass[french,a4paper]{article}
\setcounter{tocdepth}{4}
\setcounter{secnumdepth}{4}
\usepackage{booktabs}
\newcommand{\tabitem}{\textbullet~~}\title{Compte rendu de réunion}
\usepackage[bottom=2.5cm,top=2.5cm,left=2.5cm,right=2.5cm]{geometry}
\author{Noé Steiner - Alexis Marcel - Lucas Laurent - Mathias Aurand-Augier}
\date{18 Octobre 2022}
\begin{document}
\maketitle

\section*{\underline{Projet PPII - Compte rendu n°02 - réunion de suivi}}

\begin{table}[!htb]
  \centering
  \begin{tabular}{| p{7cm} | p{7cm} |}
    \hline
    \multicolumn{1}{|c|}{ Motif / type de réunion:} & \multicolumn{1}{c|}{Lieu:} \\
    \hline
    \tabitem Alexis : Présent\newline
    \tabitem Noé : Présent\newline
    \tabitem Lucas : Présent\newline
    \tabitem Mathias : Présent                      &
    \tabitem Le 18 Octobre 2022\newline
    \tabitem De 20h à 21h\newline
    \tabitem Visioconférence sur Discord                                                                   \\
    \hline
  \end{tabular}
\end{table}

\subsection*{\textit{Ordre du jour:}}

\begin{itemize}
  \item Point sur l'avancement des tâches
  \item Demande d'aide en cas de difficultés
\end{itemize}

\subsubsection*{\textit{Information échangées}}
\begin{itemize}
  \item Principaux éléments de gestion de projet terminés
  \item Intégration du livrable en \LaTeX \space en cours
  \item Liste des fonctionnalités détaillées de l'application 
\end{itemize}
\subsubsection*{\textit{Remarques / Questions}}
\begin{itemize}
  \item Comment créer le diagramme de GANTT ? Quelle application ?
\end{itemize}

\subsection*{\textit{Décisions}}


\subsection*{\textit{Actions à suivre / Todo list}}
\begin{itemize}
  \item Intégration des documents vers un document \LaTeX \space (Noé)
  \item Réalisation graphique des figures (Lucas)
  \item Rédaction de la partie gestion de projet et fonctionnalités de l'application (Alexis et Mathias)
\end{itemize}

\subsection*{\textit{Date de la prochaine réunion}}
La prochaine réunion aura lieue le Samedi 22 Octobre 2022, de 20h à 21h.

\end{document}